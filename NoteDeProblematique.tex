\documentclass[12pt]{article}
\usepackage[french]{babel}
\usepackage[utf8]{inputenc} % Required for inputting international characters
\usepackage[T1]{fontenc} % Output font encoding for international characters
\usepackage{textcomp}
\usepackage{graphicx}
\usepackage{mathpazo} % Palatino font
\usepackage{amsmath}
\usepackage{listings}
\usepackage{subcaption}
\usepackage{xcolor}
\usepackage{lscape}
\usepackage[onehalfspacing]{setspace}

\usepackage{geometry}
 \geometry{
 a4paper,
 left=35mm,
 right=35mm,
 top=40mm,
 bottom=40mm,
 }

\begin{document}
\begin{titlepage} % Suppresses displaying the page number on the title page and the subsequent page counts as page 1
	\newcommand{\HRule}{\rule{\linewidth}{0.5mm}} % Defines a new command for horizontal lines, change thickness here
	
	\center % Centre everything on the page
	
	%------------------------------------------------
	%	Headings
	%------------------------------------------------
	
	\textsc{\LARGE Ecole Centrale de Lille}\\[1cm] % Main heading such as the name of your university/college
	
	
    \begin{figure}[h!]
        \centering
        \includegraphics[width=0.6\textwidth]{logo_centrale.png}
        \label{LogoCentrale}
    \end{figure}
    
    \vfill
	
	%------------------------------------------------
	%	Title
	%------------------------------------------------
	
	\HRule\\[0.4cm]
	
	{\huge\bfseries Note de problématique: CIFRE vs non-CIFRE}\\[0.4cm] % Title of your document
	
	\HRule\\[1.0cm]
	
	%------------------------------------------------
	%	Author(s)
	%------------------------------------------------

	\begin{minipage}{0.4\textwidth}
		\begin{flushleft}
			\large
			\textit{Auteurs}\\
			E. \textsc{Le Guillou} \\
			I. \textsc{Thomas}\\
			A. \textsc{Rahier}\\
			M. \textsc{Lanvin}\\
			C. \textsc{Vo}
		\end{flushleft}
	\end{minipage}
	~
	\begin{minipage}{0.4\textwidth}
		\begin{flushright}
			\large
			\textit{Professeurs}\\
			J. P. \textsc{Richard}\\ % Supervisor's name
			C. \textsc{Belart} \\
			C. \textsc{Davy}
		\end{flushright}
	\end{minipage}

	\vfill
	
	\begin{center}
        %Nombre de mots : ?
    \end{center}
	%------------------------------------------------
	%	Date
	%------------------------------------------------
	
	\vfill\vfill % Position the date 3/4 down the remaining page
	
	{\large\today} % Date, change the \today to a set date if you want to be precise

	
	
	\vfill % Push the date up 1/4 of the remaining page
	
\end{titlepage}

\paragraph{Question initiale} %Partie d'Alban



Quand on relit attentivement les pistes de réponse de la question initiale, on se rend compte que l'on a déjà une partie de la réponse dans notre question, dans notre plan: on fait un inventaire des caractéristiques des thèses non-CIFRE et des thèses CIFRE, mais cela n'aide pas forcément à faire le choix, car il n'y a qu'une partie de la démarche.

En effet, on peut constater, lorsque l'on liste tous les axes de réponse que l'on a déjà effectué au préalable un processus d'introspection qui nous a permi d'évaluer les caractéristiques qui ont de l'importance pour nous. Par exemple, c'est parce qu'on associe une certaine importance au fait d'avoir un contact privilégié avec l'industrie qu'on se pose la question de la valeur, de l'intensité de cette caractéristique pour une thèse CIFRE ou non-CIFRE.

On décide alors de faire un pas en arrière et d'essayer d'aborder ce processus d'introspection, et de se poser la question suivante: "Comment choisir?" au sens de "Comment décider de l'importance à donner à telle ou telle caractéristique?". Nous sommes conscient que trouver la réponse à cette question est compliqué, et peut s'avérer très abstrait et très spécifique à chaque personne, mais il nous semble intéressant d'au moins tenter de décrire ce processus, afin d'aiguillonner tout futur doctorant, et de pousser la personne à suivre une telle démarche d'elle-même en préambule de leur choix.

On décide donc de concentrer une partie de nos effort sur la notion de choix elle-même, avant même d'étudier les différents types de thèse. De plus, au fur et à mesure de nos recherches initiales, nous nous sommes rendu compte que la thèse CIFRE est un label, mais pas une nécessité vis-à-vis de ses caractéristiques: une thèse CIFRE aura nécessairement certaine caractéristique, comme un contact privilégié avec l'entreprise, cependant une thèse non-CIFRE peut aussi posséder ce genre de caractéristique, ce ne sera simplement pas toujours le cas pour une thèse non-CIFRE.

Il est donc par conséquent important de ne pas s'enfermer dans l'idée que seule une thèse CIFRE correspondra à nos attentes, car certaines de ces caractéristiques peuvent être retrouvées dans une thèse non-CIFRE.

\paragraph{Question retenue} La question que nous allons retenir comme problématique de notre travail est donc la suivante: "Comment choisir entre une thèse CIFRE et une thèse non-CIFRE?". Nous allons nous concentrer sur deux axes de réponse: "Comment connaître les caractéristiques d’une thèse qui nous importent à nous?", qui va plus s'intéresser au processus d'introspection préalable au choix lui-même, et "Quelles sont les valeurs de ces caractéristiques pour une thèse CIFRE et une thèse non-CIFRE?", qui va se concentrer sur les caractéristiques des deux types de thèses.


\paragraph{Recherche documentaire}

\paragraph{Préparation des entretiens}

\end{document}
