\documentclass[12pt]{article}
\usepackage[french]{babel}
\usepackage[utf8]{inputenc} % Required for inputting international characters
\usepackage[T1]{fontenc} % Output font encoding for international characters
\usepackage{textcomp}
\usepackage{graphicx}
\usepackage{mathpazo} % Palatino font
\usepackage{amsmath}
\usepackage{listings}
\usepackage{subcaption}
\usepackage{xcolor}
\usepackage{lscape}
\usepackage[onehalfspacing]{setspace}

\usepackage{geometry}
 \geometry{
 a4paper,
 left=35mm,
 right=35mm,
 top=40mm,
 bottom=40mm,
 }

\begin{document}
\begin{titlepage} % Suppresses displaying the page number on the title page and the subsequent page counts as page 1
	\newcommand{\HRule}{\rule{\linewidth}{0.5mm}} % Defines a new command for horizontal lines, change thickness here
	
	\center % Centre everything on the page
	
	%------------------------------------------------
	%	Headings
	%------------------------------------------------
	
	\textsc{\LARGE Ecole Centrale de Lille}\\[1cm] % Main heading such as the name of your university/college
	
	
    \begin{figure}[h!]
        \centering
        \includegraphics[width=0.6\textwidth]{logo_centrale.png}
        \label{LogoCentrale}
    \end{figure}
    
    \vfill
	
	%------------------------------------------------
	%	Title
	%------------------------------------------------
	
	\HRule\\[0.4cm]
	
	{\huge\bfseries Note de problématique: CIFRE vs non-CIFRE}\\[0.4cm] % Title of your document
	
	\HRule\\[1.0cm]
	
	%------------------------------------------------
	%	Author(s)
	%------------------------------------------------

	\begin{minipage}{0.4\textwidth}
		\begin{flushleft}
			\large
			\textit{Auteurs}\\
			E. \textsc{Le Guillou} \\
			I. \textsc{Thomas}\\
			A. \textsc{Rahier}\\
			M. \textsc{Lanvin}\\
			C. \textsc{Vo}
		\end{flushleft}
	\end{minipage}
	~
	\begin{minipage}{0.4\textwidth}
		\begin{flushright}
			\large
			\textit{Professeurs}\\
			J. P. \textsc{Richard}\\ % Supervisor's name
			C. \textsc{Belart} \\
			C. \textsc{Davy}
		\end{flushright}
	\end{minipage}

	\vfill
	
	\begin{center}
        %Nombre de mots : ?
    \end{center}
	%------------------------------------------------
	%	Date
	%------------------------------------------------
	
	\vfill\vfill % Position the date 3/4 down the remaining page
	
	{\large\today} % Date, change the \today to a set date if you want to be precise

	
	
	\vfill % Push the date up 1/4 of the remaining page
	
\end{titlepage}

\paragraph{Question initiale} %Partie d'Alban

Au fur et à mesure de nos recherches initiales, nous nous sommes rendu compte que la thèse CIFRE est un label, mais pas une nécessité vis-à-vis de ses caractéristiques: une thèse CIFRE aura nécessairement certaines caractéristiques, comme un contact privilégié avec l'entreprise, cependant une thèse non-CIFRE comme le contrat doctorale peut aussi posséder ce genre de caractéristique, ce ne sera simplement pas toujours le cas.

Il est donc par conséquent important de ne pas s'enfermer dans l'idée que seule une thèse CIFRE correspondra à nos attentes, car certaines de ces caractéristiques peuvent être retrouvées dans une thèse non-CIFRE. Pour refléter ce fait, nous choisissons de modifier notre problématique afin que celle-ci englobe de manière évidente à la fois les thèses CIFRE et les contrats doctoraux, dont nous donnerons une définition plus précise dans la suite du texte.

\paragraph{Question retenue} La question que nous allons retenir comme problématique de notre travail est donc la suivante: " Comment choisir les modalités de sa thèse ?". Pour répondre à cette question, nous articulerons notre réflexion autour d'un objectif de carrière dans le privée ainsi que d'un objectif de carrière dans le public. L'objectif de bien-être dans la thèse sera aussi pris en compte.

\paragraph{Recherche documentaire}

\paragraph{Préparation des entretiens}

\end{document}
