\documentclass[12pt]{article}
\usepackage[french]{babel}
\usepackage[utf8]{inputenc} % Required for inputting international characters
\usepackage[T1]{fontenc} % Output font encoding for international characters
\usepackage{textcomp}
\usepackage{graphicx}
\usepackage{mathpazo} % Palatino font
\usepackage{amsmath}
\usepackage{listings}
\usepackage{subcaption}
\usepackage{xcolor}
\usepackage{lscape}
\usepackage[onehalfspacing]{setspace}

\usepackage{geometry}
 \geometry{
 a4paper,
 left=35mm,
 right=35mm,
 top=40mm,
 bottom=40mm,
 }

\begin{document}
\begin{titlepage} % Suppresses displaying the page number on the title page and the subsequent page counts as page 1
	\newcommand{\HRule}{\rule{\linewidth}{0.5mm}} % Defines a new command for horizontal lines, change thickness here
	
	\center % Centre everything on the page
	
	%------------------------------------------------
	%	Headings
	%------------------------------------------------
	
	\textsc{\LARGE Ecole Centrale de Lille}\\[1cm] % Main heading such as the name of your university/college
	
	
    \begin{figure}[h!]
        \centering
        \includegraphics[width=0.6\textwidth]{logo_centrale.png}
        \label{LogoCentrale}
    \end{figure}
    
    \vfill
	
	%------------------------------------------------
	%	Title
	%------------------------------------------------
	
	\HRule\\[0.4cm]
	
	{\huge\bfseries Note de problématique: CIFRE vs non-CIFRE}\\[0.4cm] % Title of your document
	
	\HRule\\[1.0cm]
	
	%------------------------------------------------
	%	Author(s)
	%------------------------------------------------

	\begin{minipage}{0.4\textwidth}
		\begin{flushleft}
			\large
			\textit{Auteurs}\\
			E. \textsc{Le Guillou} \\
			I. \textsc{Thomas}\\
			A. \textsc{Rahier}\\
			M. \textsc{Lanvin}\\
			C. \textsc{Vo}
		\end{flushleft}
	\end{minipage}
	~
	\begin{minipage}{0.4\textwidth}
		\begin{flushright}
			\large
			\textit{Professeurs}\\
			J. P. \textsc{Richard}\\ % Supervisor's name
			C. \textsc{Belart} \\
			C. \textsc{Davy}
		\end{flushright}
	\end{minipage}

	\vfill
	
	\begin{center}
        %Nombre de mots : ?
    \end{center}
	%------------------------------------------------
	%	Date
	%------------------------------------------------
	
	\vfill\vfill % Position the date 3/4 down the remaining page
	
	{\large\today} % Date, change the \today to a set date if you want to be precise

	
	
	\vfill % Push the date up 1/4 of the remaining page
	
\end{titlepage}

\paragraph{Question initiale} %Partie d'Alban

Au fur et à mesure de nos recherches initiales, nous nous sommes rendu compte que la thèse CIFRE est un label, mais pas une nécessité vis-à-vis de ses caractéristiques: une thèse CIFRE aura nécessairement certaines caractéristiques, comme un contact privilégié avec l'entreprise, cependant une thèse non-CIFRE comme le contrat doctorale peut aussi posséder ce genre de caractéristique, ce ne sera simplement pas toujours le cas.

Il est donc par conséquent important de ne pas s'enfermer dans l'idée que seule une thèse CIFRE correspondra à nos attentes, car certaines de ces caractéristiques peuvent être retrouvées dans une thèse non-CIFRE. Pour refléter ce fait, nous choisissons de modifier notre problématique afin que celle-ci englobe de manière évidente à la fois les thèses CIFRE et les contrats doctoraux, dont nous donnerons une définition plus précise dans la suite du texte.

\paragraph{Question retenue} La question que nous allons retenir comme problématique de notre travail est donc la suivante: " Comment choisir les modalités de sa thèse ?". Pour répondre à cette question, nous articulerons notre réflexion autour d'un objectif de carrière dans le privée ainsi que d'un objectif de carrière dans le public. L'objectif de bien-être dans la thèse sera aussi pris en compte.

\paragraph{Recherche documentaire}

\paragraph{Préparation des entretiens}

Pour répondre à notre questionnement, nous allons réaliser des entretiens semi-directifs. En effet, nous allons interroger des personnes du terrain, ce qui nous permettra d'avoir un bon aperçu de la réalité et dépasser nos croyances. Jusque là les recherches bibliographiques nous ont permis d'établir des hypothèses. Ensuite nous avons établit une grille d'entretien pemerttant d'aborder les sujets en question tout en prenant garde à éviter les déformations liées à l'exercice de l'entretien. Parmi les effets indésirables de l'entretien, nous avons découvert les effets de cadrage et de halo. Le premier consiste à influence la réponse de la personne interrogée par la formulation de la question posée. Le deuxième correspond au mécanisme où des questions trop similaires induisent des réponses qui le seront également. De plus, il est nécessaire de garder à l'esprit certains biais. Le biais de désirabilité amène notamment les personnes interrogées à présenter une position sociale enviable. Cela peut se traduire par des distorsion dans leur discours, il faudra donc y être vigiliant au plus possible.

Nous avons décidé de réaliser une quinzaine d'entretiens. Parmi eux deux ou trois entretiens seront des pré-tests de notre grille d'entretien finale. Cela nous permettra de percevoir les questions qui n'apportent pas les réponses souhaitées ou bien de corriger les éventuels problèmes de clarté ou d'imprécision de certaines questions. Pour ce qui est du public visé, nous soushaitons interroger une majorité de personnes ayant récemment réalisé une thèse (moins de trois ans). En effet, nous pensons que ce public nous apportera des réponses plus proches de celles qui nous tiennent à coeur. Nous avons aussi prévu d'interroger des personnes avec une carrière plus avancée, pour recueillir leur témoignage sur l'influence de la thèse dans la carrière professionnelle sur le long terme. A partir des hypothèses établies a priori à partir de nos croyances, des recherches documentaires réalisées et des résultats des entretiens nous pourront ainsi analyser les écarts entre les deux et obtenir une réponse à nos questions qui sera plus proche de la réalité.

\end{document}
